\documentclass[review]{elsarticle}
\usepackage{amsmath}
\usepackage{hyperref}
\usepackage{tikz}
\usetikzlibrary{matrix}

\begin{document}

\begin{frontmatter}

\title{Cross Matrix with Vectors}

\begin{abstract}
This is a sample of a cross matrix with matrices, a column vector, and a row vector.
\end{abstract}

\end{frontmatter}




By introducing new coefficients $\tilde{\alpha_{-1}}$ and
 enhancing the retention of results from previous one-step RKC 
 calculations, the computational overhead remains almost unchanged 
 when compared to the two-step RKC method. However, the inclusion 
 of $\tilde{\alpha_{-1}}$ grants us extra maneuverability for
  adjusting $\omega_{1}$. Consequently, the coefficients
   $\eta$, $\omega_{0}$, and $\omega_{1}$ in the NTRKC
    scheme exhibit slight deviations from those of the 
    one-step RKC method. In the context of NTRKC, the
     modulation of the damping parameter $\varepsilon$ 
     and $\omega_{1}$ serves to shape the stabilizing 
     domain. Higher values of $\varepsilon$ and $\omega_{1}$ entail a 
     trade-off between domain width and length, favoring a wider 
     region at the expense of domain length. Conversely, lower 
     values of $\varepsilon$ and $\omega_{1}$ lead to a more
      elongated yet narrower stabilizing domain. Notably, 
      NTRKC demonstrates superior stabilizing domain
       characteristics when compared to the two-step RKC method, yielding 
       heightened numerical accuracy. In the subsequent section 
       dedicated to stabilizing domain analysis, we will delve further 
       into this topic, providing insights and presenting two distinct
        sets of coefficients.
\end{document}